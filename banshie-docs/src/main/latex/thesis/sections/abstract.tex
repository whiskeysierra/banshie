\newpage
% no idea why it's necessary to explicitly switch to english, but it won't work otherwise
\selectlanguage{english}
\begin{abstract}
Within the age of the Internet and social media sites there is a vast amount of mainly unstructured data being produced on a daily basis. Way too much to handle it in a manual fashion. \textit{\gls{IE}} aims to define, develop and test techniques to extract information from unstructured or semi-structured data sources and transform them into a representation better suited for further analysis. Critically evaluating proposed information extraction algorithms is crucial to ensure a continuous improvement in system performances. 

This thesis aims to contribute to the process of automation and standardization of the IE evaluation process. It does so by comparing different IE tasks and approaches for evaluating corresponding systems. The findings were then implemented into a modular framework which does not only support IE performance analysis but also provide benchmarking capabilities to measure runtime performances. The practicability of the framework will be shown by implemented adapters for different Open Source extraction systems and perform evaluations.
\end{abstract}

\newpage
% german abstract
\selectlanguage{ngerman}
\begin{abstract}
Im Zeitalter des Internets und Social-Media-Seiten werden täglich riesige Mengen von unstrukturiereten Daten produziert. Wesentlich mehr, als sich manuell noch weiterverarbeiten geschweige denn erfassen lassen. \textit{Information Extraction} versucht Techniken und Algorithmen zu definieren, entwickeln und zu testen um Informationen aus unstrukturierten oder semistrukturierten Datenquellen zu extrahieren und in eine Darstellung zu überführen die für eine spätere Analyse und Weiterverarbeit besser geeignet ist. Die kritische Bewertung von IE-Algorithmen ist entscheidend, um eine kontinuierliche Verbesserung der Systemleistung zu gewährleisten.

Diese These zielt darauf ab, zur Automatisierung und Standardisierung des IE-Evaluationsprozesses beitragen. Dazu werden verschiedene IE-Aufgaben und Ansätze für die Beurteilung entsprechender Systeme aufgezeigt und beurteilt. Die Ergebnisse werden dann in Form eines modularen Framework umgesetzt, das nicht nur die Extraktionsleistung von IE-Systemen, sondern auch das Laufzeitverhalten bewertet. Die Praxistauglichkeit des Frameworks wird anhand der Integration und Evaluation von Open Source Extraktoren gezeigt.
\end{abstract}

% switch back to english
\selectlanguage{english}