\section{Information Extraction}
\label{sec:information-extraction}

\fxfatal{Information Extraction}

\subsection{Definition}

\begin{quote}
Information Extraction is a technology that is futuristic from the user's point of view in the current information-driven world. Rather than indicating which documents need to be read by a user, it extracts pieces of information that are salient to the user's needs. Links between the extracted information and the original documents are maintained to allow the user to reference context. The kinds of information that systems extract vary in detail and reliability.

\hfill \textbf{Message Understanding Conference (MUC)}

\hfill \citeauthor{Chinchor:2001} \cite{Chinchor:2001}
\end{quote}

\begin{quote}
Information Extraction refers to the automatic extraction of structured information such as entities, relationships between entities, and attributes describing entities from unstructured sources. This enables much richer forms of queries on the abundant unstructured sources than possible with keyword searches alone. When structured and unstructured data co-exist, information extraction makes it possible to integrate the two types of sources and pose queries spanning them.

\hfill \textbf{Information Extraction}

\hfill \citeauthor{Sarawagi:2008} \cite{Sarawagi:2008}
\end{quote}

\begin{quote}
Information extraction (IE) is the task of automatically extracting structured information from unstructured and/or semi-structured machine-readable documents. In most of the cases this activity concerns processing human language texts by means of natural language processing (NLP). Recent activities in multimedia document processing like automatic annotation and content extraction out of images/audio/video could be seen as information extraction.

\hfill \textbf{Information extraction}

\hfill \citeauthor{Wikipedia:IE:2012} \cite{Wikipedia:IE:2012}
\end{quote}

\newpage
\subsection{History}

\newpage
\subsection{Most typical tasks}

\subsubsection{Named Entity Recognition}

\subsubsection{Conference Resolution}

\subsubsection{Template Element Construction}

\subsubsection{Template Relation Construction}
or Relationship extraction

\subsubsection{Scenario Template Production}

\subsubsection{Restoring information structure}

\newpage
\subsection{Progression and development}
towards less domain-specific ie tools

\subsubsection{Knowledge Engineering}

\subsubsection{Machine Learning}

\subsubsection{Wrapper Generation}

\subsubsection{Automatic Content Extraction}

\subsubsection{\acrlong{OBIE}}
\gls{OBIE}

\newpage
\subsection{Related fields}

\subsubsection{Information Retrieval}

\subsubsection{Natural Language Processing}
\gls{NLP} ...

\subsubsection{Machine Learning}
