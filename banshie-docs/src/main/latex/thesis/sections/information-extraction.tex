\section{Information Extraction}
\label{sec:information-extraction}

\fxfatal{Information Extraction}
\fxfatal{Subfield of NLP}

\subsection{Definition}

\begin{quote}
Information Extraction is a technology that is futuristic from the user's point of view in the current information-driven world. Rather than indicating which documents need to be read by a user, it extracts pieces of information that are salient to the user's needs. Links between the extracted information and the original documents are maintained to allow the user to reference context. The kinds of information that systems extract vary in detail and reliability.

\hfill \textbf{Message Understanding Conference (MUC)}

\hfill \citeauthor{Chinchor:2001} \cite{Chinchor:2001}
\end{quote}

\begin{quote}
Information Extraction refers to the automatic extraction of structured information such as entities, relationships between entities, and attributes describing entities from unstructured sources. This enables much richer forms of queries on the abundant unstructured sources than possible with keyword searches alone. When structured and unstructured data co-exist, information extraction makes it possible to integrate the two types of sources and pose queries spanning them.

\hfill \textbf{Information Extraction}

\hfill \citeauthor{Sarawagi:2008} \cite{Sarawagi:2008}
\end{quote}

\begin{quote}
Information extraction (IE) is the task of automatically extracting structured information from unstructured and/or semi-structured machine-readable documents. In most of the cases this activity concerns processing human language texts by means of natural language processing (NLP). Recent activities in multimedia document processing like automatic annotation and content extraction out of images/audio/video could be seen as information extraction.

\hfill \textbf{Information extraction}

\hfill \citeauthor{Wikipedia:IE:2012} \cite{Wikipedia:IE:2012}
\end{quote}

\newpage
\subsection{History}
The area of text understanding can be considered as the basis for the \gls{IE}. In this regard, researchers studied methods in the field of Artificial Intelligence, which reproduce the contents of a text in exact form \cite{Siefkes:2007}\cite{Eikvil:1999}. The first application of information extraction occurred in the 1950s, were  the information from texts  were reduced into a table structure. Sager 1981 published works in the field of medicine and used manually-created structures and templates. This "information formats" were obtained based on rules. Complex system developments were made by Hayes et al. in 1992 \cite{Grishman:1997}\cite{Gaizauskas:1998}\cite{Wilks:1997}.

The increase in research in the field of IE forced the \gls{DARPA} in the late 1980s to initiate an operation. Thus, the \gls{MUC} have been launched, aimed at competing implementation and evaluation of IE systems. Participants received test data of a particular domain and a special output format. They than developed IE systems based on these requirements, their performances were compared in terms of unknown documents at conferences. Manually created templates were used as reference data \cite{Grishman:1996}\cite{Grishman:1997}.

The conferences were held between 1987 and 1998. The following table lists the domains and the number of training and reference documents of the respective conferences \cite{Turmo:2006}\cite{Appelt:1999}\cite{Cunningham:2005}:

\begin{table}[H]
\centering
\begin{tabular*}{\textwidth}{ l l l l l l }
	\toprule
	& Year & Topic & \shortstack{Number of \\ systems} & \shortstack{Traning \\ documents} & \shortstack{Reference \\ documents} \\
	\midrule
	MUC-1 & 1987 & Marine operations & 6 & 12 & 2 \\
	MUC-2 & 1989 & Marine operations & 8 & 105 & 25 \\
	MUC-3 & 1991 & Terror acts & 15 & 1300 & 300 \\
	MUC-4 & 1992 & Terror acts & 17 & - & - \\
	MUC-5 & 1993 & Joint ventures, & 17 & - & - \\
	& & microelectronics & & & \\
	MUC-6 & 1995 & Management & 17 & - & - \\
	& & changes & & & \\
	MUC-7 & 1998 & Space travel & - & - & - \\
	\bottomrule
\end{tabular*}
\caption{Message Understanding Conferences}
\end{table}

The conferences have made a decisive contribution to the development of information extraction. On one hand, the formulation of sub-tasks and metrics should be noted and on the other hand, the striving for domain independence and portability of \gls{IE} systems. The meeting of various research groups and the implementation of systems based on the same task offers enormous opportunities to exchange ideas and to overcome theoretical and paradigmatic differences \cite{Cimiano:2003}\cite{Lehnert:1994}.

\newpage
\subsection{Most typical tasks}

The Message Understanding Conferences structures the information extraction into the following sub-tasks due to its complexity \cite{Carstensen:2010}\cite{Lavelli:2008}:

The \gls{IE} sub-tasks will be explained using the following example document:

\begin{quote}
The shiny red rocket was fired on Tuesday. It is the brainchild of Dr. Big Head. Dr. Head is a staff scientist at We Build Rockets Inc.
\cite{Cunningham:2005}
\end{quote}

\subsubsection{Named Entity Recognition}
\gls{NER}, also referred to as Name Recognition, Entity Identification or Entity Extraction, is defined as the extraction of known entity names. These include people, organizations, locations, products, date/times and certain numerical expressions \cite{Linsmayr:2010}. 

\begin{table}[H]
\centering
\begin{tabular*}{\textwidth}{ l  l }
	\toprule
	\textbf{Type} & \textbf{Value} \\
	\midrule
	\texttt{PRODUCT} & rocket \\
	\texttt{DATE} & Tuesday \\
	\texttt{PERSON} & Dr. Big Head \\
	\texttt{ORGANIZATION} & We Build Rockets Inc. \\
	\bottomrule
\end{tabular*}
\caption{Named Entity Recognition example output}
\end{table}

\subsubsection{Coreference Resolution}
\gls{CO}, also referred to as Coreference Analysis, Deduplication or Record Linkage. As entities and relationships are extracted from the unstructured source, they need to be integrated with existing databases and with repeated mentions of the same information in the unstructured source. The main challenge in this task is deciding if two strings refer to the same entity in spite of the many noisy variants in which it appears in the unstructured source \cite{Sarawagi:2008}.

Example: \textit{It} in \enquote{It is the brainchild of Dr. Big Head. Dr. Head is a staff scientist at We Build Rockets Inc.} refers to the previously extracted entity \textit{rocket}.

\subsubsection{Template Element Construction}
\gls{TE}, also referred to as Attribute Extraction, describes the task to associate a given entity with the value of an adjective describing the entity. The value of this adjective typically needs to be derived by combining soft clues spread over many different words around the entity
 \cite{Sarawagi:2008}.

\begin{table}[H]
\centering
\begin{tabular*}{\textwidth}{ l  l }
	\toprule
	\textbf{Attribute} & \textbf{Target} \\
	\midrule
	shiny red & rocket \\
	brainchild of Dr. Big Head & rocket \\
	\bottomrule
\end{tabular*}
\caption{Template Element Construction example output}
\end{table}

\subsubsection{Template Relation Construction}
\gls{TR}, also referred to as Relationship extraction, defines to task of extracting relationship information of previously extracted entities. Relationships are defined over two or more entities related in a predefined way. Examples are \enquote{is employee of} relationship between a person and an organization or \enquote{is acquired by} relationship between pairs of companies \cite{Sarawagi:2008}.

The extraction of relationships differs from the extraction of entities in one significant way. Whereas entities refer to a sequence of words in the source and can be expressed as annotations on the source, relationships are not annotations on a subset of words. Instead they express the associations between two separate text snippets representing the entities \cite{Sarawagi:2008}.

\begin{table}[H]
\centering
\begin{tabular*}{\textwidth}{ l l l l l }
	\toprule
	\textbf{Entity} & \textbf{Relation}  & \textbf{Entity} \\
	\midrule
	Dr. Big Head & works for & We Build Rockets Inc. \\
	\bottomrule
\end{tabular*}
\caption{Template Element Construction example output}
\end{table}

\newpage
\subsubsection{Scenario Template Production}
\gls{ST}, also referred to as Event Extraction, tries to extract events that previously extracted entities participate in \cite{Cunningham:2005}.

Regarding the given example document, \gls{ST} discovers that there was a rocket launching event in which the various entities were involved \cite{Cunningham:2005}.

\subsubsection{Restoring information structure such as Lists, Tables and Ontologies}
The scope of extraction systems has now expanded to include the extraction of not such atomic entities and flat records but also richer structures such as tables, lists, and trees from various types of documents \cite{Sarawagi:2008}.

\newpage
\subsection{Progression and development}
towards less domain-specific ie tools

\subsubsection{Knowledge Engineering}

\subsubsection{Machine Learning}

\subsubsection{Wrapper Generation}

\subsubsection{Automatic Content Extraction}

\subsubsection{\acrlong{OBIE}}
\gls{OBIE}

\newpage
\subsection{Related fields}

\subsubsection{Information Retrieval}

\subsubsection{Natural Language Processing}
\gls{NLP} ...

\subsubsection{Machine Learning}
