\section{Information Extraction}
\label{sec:information-extraction}

\fxfatal{Information Extraction}
\fxfatal{Subfield of NLP}

\subsection{Definition}

\begin{quote}
Information Extraction is a technology that is futuristic from the user's point of view in the current information-driven world. Rather than indicating which documents need to be read by a user, it extracts pieces of information that are salient to the user's needs. Links between the extracted information and the original documents are maintained to allow the user to reference context. The kinds of information that systems extract vary in detail and reliability.

\hfill \textbf{Message Understanding Conference (MUC)}

\hfill \citeauthor{Chinchor:2001} \cite{Chinchor:2001}
\end{quote}

\begin{quote}
Information Extraction refers to the automatic extraction of structured information such as entities, relationships between entities, and attributes describing entities from unstructured sources. This enables much richer forms of queries on the abundant unstructured sources than possible with keyword searches alone. When structured and unstructured data co-exist, information extraction makes it possible to integrate the two types of sources and pose queries spanning them.

\hfill \textbf{Information Extraction}

\hfill \citeauthor{Sarawagi:2008} \cite{Sarawagi:2008}
\end{quote}

\begin{quote}
Information extraction (IE) is the task of automatically extracting structured information from unstructured and/or semi-structured machine-readable documents. In most of the cases this activity concerns processing human language texts by means of natural language processing (NLP). Recent activities in multimedia document processing like automatic annotation and content extraction out of images/audio/video could be seen as information extraction.

\hfill \textbf{Information extraction}

\hfill \citeauthor{Wikipedia:IE:2012} \cite{Wikipedia:IE:2012}
\end{quote}

\newpage
\subsection{History}
The area of text understanding can be considered as the basis for the \gls{IE}. In this regard, researchers studied methods in the field of Artificial Intelligence, which reproduce the contents of a text in exact form. The first application of information extraction occurred in the 1950s, were  the information from texts  were reduced into a table structure. Sager 1981 published works in the field of medicine and used manually-created structures and templates. This "information formats" were obtained based on rules. Complex system developments were made by Hayes et al. in 1992.

The increase in research in the field of IE forced the DARPA (Defense Advanced Research Projects Agency) in the late 1980s to initiate an operation. Thus, the "Message Understanding Conferences" (MUC) have been launched, aimed at competing implementation and evaluation of IE systems. Participants received test data of a particular domain and a special output format. They than developed IE systems based on these requirements, their performances were compared in terms of unknown documents at conferences. Manually created templates were used as reference data.

The conferences were held between 1987 and 1998. The following table lists the domains and the number of training and reference documents of the respective conferences:

\begin{table}[H]
\centering
\begin{tabular*}{\textwidth}{ l l l l l l }
	\toprule
	& Year & Topic & \shortstack{Number of \\ systems} & \shortstack{Traning \\ documents} & \shortstack{Reference \\ documents} \\
	\midrule
	MUC-1 & 1987 & Marine operations & 6 & 12 & 2 \\
	MUC-2 & 1989 & Marine operations & 8 & 105 & 25 \\
	MUC-3 & 1991 & Terror acts & 15 & 1300 & 300 \\
	MUC-4 & 1992 & Terror acts & 17 & - & - \\
	MUC-5 & 1993 & Joint ventures, & 17 & - & - \\
	& & microelectronics & & & \\
	MUC-6 & 1995 & Management & 17 & - & - \\
	& & changes & & & \\
	MUC-7 & 1998 & Space travel & - & - & - \\
	\bottomrule
\end{tabular*}
\caption{Message Understanding Conferences}
\end{table}

The conferences have made a decisive contribution to the development of information extraction. On one hand, the formulation of sub-tasks and metrics should be noted and on the other hand, the striving for domain independence and portability of \gls{IE} systems. The meeting of various research groups and the implementation of systems based on the same task offers enormous opportunities to exchange ideas and to overcome theoretical and paradigmatic differences.

\newpage
\subsection{Most typical tasks}

The Message Understanding Conferences structures the information extraction into the following sub-tasks due to its complexity.

\fxfatal{Example text}

\begin{quote}
Sam Schwartz retired as executive vice president of the famous hot dog manufacturer, Hupplewhite Inc. He will be succeded by Harry Himmelfarb.
\cite{Grishman:1997}
\end{quote}

\subsubsection{Named Entity Recognition}
\gls{NER}, also referred to as Name Recognition, Entity Identification or Entity Extraction, is defined as the extraction of known entity names. These include people, organizations, locations, date/time and certain numerical expressions. 

\begin{table}[H]
\centering
\begin{tabular*}{\textwidth}{ l  l }
	\toprule
	\textbf{Type} & \textbf{Value} \\
	\midrule
	\texttt{PERSON} & Sam Schwartz \\
	\texttt{ORGANIZATION} & Hupplewhite Inc. \\
	\texttt{PERSON} & Harry Himmelfarb \\
	\bottomrule
\end{tabular*}
\caption{Named Entity Recognition example output}
\end{table}

\subsubsection{Coreference Resolution}
\gls{CO}, also referred to as Coreference Analysis, ...

Example: \textit{He} in \enquote{He will be succeded by Harry Himmelfarb.} refers to the previously extracted entity \textit{Sam Schwatz}.

\subsubsection{Template Element Construction}
\gls{TE}, also referred to as Attribute Extraction, ...

\begin{table}[H]
\centering
\begin{tabular*}{\textwidth}{ l  l }
	\toprule
	\textbf{Attribute} & \textbf{Target} \\
	\midrule
	executive vice president & Sam Schwartz \\
	famous hot dog manufacturer & Hupplewhite Inc. \\
	\bottomrule
\end{tabular*}
\caption{Template Element Construction example output}
\end{table}

\subsubsection{Template Relation Construction}
\gls{TR}, also referred to as Relationship extraction, ...

\begin{table}[H]
\centering
\begin{tabular*}{\textwidth}{ l l l l l }
	\toprule
	\textbf{Value} & \textbf{Relation}  & \textbf{Value} \\
	\midrule
	Sam Schwartz & worked for & Hupplewhite Inc. \\
	Hupplewhite Inc. & has & executive vice president \\
	Harry Himmelfarb & works for & Hupplewhite Inc. \\
	\bottomrule
\end{tabular*}
\caption{Template Element Construction example output}
\end{table}

\subsubsection{Scenario Template Production}
\gls{ST}, ...

Identification of two persons, a retiring executive vice president and a new one which takes over the job, and a company which employs them.

\subsubsection{Restoring information structure}

\newpage
\subsection{Progression and development}
towards less domain-specific ie tools

\subsubsection{Knowledge Engineering}

\subsubsection{Machine Learning}

\subsubsection{Wrapper Generation}

\subsubsection{Automatic Content Extraction}

\subsubsection{\acrlong{OBIE}}
\gls{OBIE}

\newpage
\subsection{Related fields}

\subsubsection{Information Retrieval}

\subsubsection{Natural Language Processing}
\gls{NLP} ...

\subsubsection{Machine Learning}
