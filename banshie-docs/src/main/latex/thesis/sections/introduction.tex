\pagenumbering{arabic}
\section{Introduction}

\subsection{Background and motivation}
Within the age of the internet and social media sites there is a vast amount of mainly unstructured data being produced on a daily basis. Way too much to handle it in a manual fashion. A lot of research has been done to define, develop and test techniques to extract information from unstructured or semi-structured data sources and transform them into a representation better suited for further analysis. This scientific subfield of Computer Science is called \gls{IE}.

Since the beginning of \gls{IE} evaluating the quality of an extractor was always an important factor. But \gls{IE} is missing two things: a set of comprehensive, standard evaluation measures and a well-designed, extensible evaluation framework. Most of the evaluation measures used in current tools are lent from \textit{Information Retrieval}, which usually don't really grasp the inexact nature of \gls{IE}.

\subsection{Objective}
The goal of this thesis is a formal discussion of known and used performance measures for IE and a working prototype of a highly modular benchmark framework for Java-based platforms to run and test information extraction systems in isolation to measure IE-related performance measures, e.g. \textit{precision}, \textit{recall} and \textit{F-Measure}, as well as runtime performance measures, e.g. cpu time and memory consumption.

\subsection{Structure}
The background knowledge required to put this thesis into context is separated into three chapters: \nameref{sec:information-extraction}, \nameref{sec:evaluation-methodology} and \nameref{sec:modularity}:

Chapter \ref{sec:information-extraction} (\nameref{sec:information-extraction}) contains different definitions of \gls{IE}, a small discourse about its history, a more or less complete list of the most typical tasks in \gls{IE} and some information about common \gls{IE} approaches, current developments and related fields. \nameref{sec:evaluation-methodology}, chapter \ref{sec:evaluation-methodology}, shows and discusses current state-of-the-art evaluation techniques and performance measures for information extraction systems and tools. Chapter \ref{sec:modularity} (\nameref{sec:modularity}) contains different definitions, goals and requirements of modularity as well as a quick overview about modularity in general and Java and the \gls{OSGi} service platform in particular.

The chapter \ref{sec:design} \nameref{sec:design} describes the framework requirements, architecture and implementation steps. The conclusion in chapter \ref{sec:conclusion} will be a critical review of the work done in the course of this thesis as well as an outlook on future work.